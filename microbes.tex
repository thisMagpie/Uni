\documentclass[12pt]{article}
\usepackage{natbib}
\usepackage{todonotes}
\usepackage[hmargin=1.5cm,vmargin=1.5cm]{geometry}

\usepackage{hyperref}
\usepackage{cleveref}
\crefname{table}{table}{tables}
\Crefname{table}{Table}{Tables}
\crefname{figure}{figure}{figures}
\Crefname{figure}{Figure}{Figures}
\crefname{equation}{equation}{equations}
\Crefname{Equation}{Equation}{Equation}

\title{A Review of The Study of Microbes in Space}
\author{Andrew Coop, Edvinas Pakanavičius, Emma Ryan, Freddy Tanner, Magdalen Berns}
\date{\today}

%\setlength\parindent{0pt}
\begin{document}
\maketitle
\thispagestyle{empty}

\begin{abstract}
\noindent

\end{abstract}

\clearpage
\tableofcontents
\thispagestyle{empty}
\clearpage

\section{Introduction}
\section{Emma's Bit}
\subsection{What Are Microbes?}
Microbes are defined as living organisms that are too small to be seen with the naked eye. Most are single celled but they can be multicellular too. There are four different types of microorganisms, bacteria, algae fungi and protis , and within these groups there is large diversity. Microbes metabolise in a variety of ways, such as anaerobically, via fermentation or lithotrophy.  Microbes are an integral part of the nitrogen cycle, the carbon cycle and producing oxygen; all of which help maintain the Earth?s atmosphere . Without microbes, the planet would be a very different place.
\\
\\
Microbes have inhabited the Earth for over 3 billion years . They were the first forms of life on the planet and, for most of the Earth?s history, the only form of life on the planet. They have been found in extremes of the Earth?s environment, such as under the sea ice in Antarctica , the Mariana Trench in the Pacific Ocean  and Mammoth Hot Springs in Yellowstone Park, as they adapt easily to their surroundings. 
\\
\\
Microbes adapt easily because they can evolve at a fast rate due to their simple structure and the nature of their reproduction. Microbes can evolve asexually and sexually. Methods of asexual reproduction are binary fission, mitosis, budding, spore formation, cytokinesis and schizogony . Although bacteria do not have nuclei, they can be considered to reproduce sexually in a process called conjugation . Genetic material is transferred between two cells using a pilus. This means that bacteria can adapt quickly to new environments as desirable traits can be passed along to different cells. It is thought that this method is the key factor for the survival ability of microorganisms.  
\\
\\Many of the aspects of the structure of microorganisms are thought to be because of the effects of gravity. The cytoskeleton is responsible for the maintaining the shape of a cell and the support and movement of a cell. The support that a cytoskeleton provides is necessary against a cell?s weight.  Flagellae are a common feature of microorganisms, allowing cell motility. Being able to propel themselves and control their position means it is possible to reach nutrients and safer environments.  The gravity on Earth has also led to the evolution of gravity receptors. Knowing the acceleration of prey or predators is important for many living cells as it could be essential for survival.   
\\
\\Microbes have many different uses on Earth. Microbes can be used to make wine, beer and whiskey in the alcohol industry, cheese, coffee and bread in the food industry and antibiotics, vitamins and insulin in the medical industry . It is believed that microbes will be able to be used in space, for future ventures such as biomining , life support machines  and biofuels.  However, microbes will not act the same in space as they do on Earth, due to the different conditions, and their behaviour in microgravity must be studied before fully understanding their functionality. 
\\
\\Microgravity, as the name suggests, is when the gravitational force experienced by a body is small . It can be achieved in two different ways. The first is not being close to any large bodies of mass which have a large gravitational attraction (such as the Earth) and the second is being in free fall. 

\subsection{Why Should the Effects of Microgravity on Microbes be Studied?}
Studying the effects of microgravity on microbes has benefits on Earth and in space. It will aid academic learning, help with medical advances and developments and could contribute key ideas and technology to space travel and exploration.
\\ \\It can give a better understanding of the evolution of life on Earth. Microbes are the earliest life form on Earth and by learning about their behaviour in microgravity may help explain why they evolved as they did under the gravitational conditions on Earth.
\\ \\All living creatures have evolved under the effects of gravity.  The skeleton of land dwelling creatures is to support the mass against the effects of gravity. The size of animals is restricted by gravity. Many of the internal processes of terrestrial life, such as circulation, have had to work against the effects of gravity as blood is needed to pump all around the body. By understanding how gravity affects simple cells, like microorganisms, it can lead to a better knowledge of more complicated life forms and how the cells within the body work.
\\ \\It can also give an indication of possible forms of extra-terrestrial life. If extra-terrestrial life does exist, it is likely that they have evolved from microorganisms too as these are the simplest forms of life and that they have not experienced the same gravity as there is on Earth.
\\ \\There are also practical uses to studying microbes in space. One contemporary issue are ?super viruses? . This is the resistance that microbes and viruses have against modern medicine because of their evolutionary prowess. Studying the effects of microgravity will allow insight into the genetic basis of infection. Any change in behaviour could help teach how pathogens evolve and cause disease. This could lead to advances in immunity against infections and viruses.
\\ \\Microgravity provides a completely different environment from what is typically experienced on Earth. It has led to developments in biotechnology. More efficient ways of microencapsulation, a method of delivering medicine to parts of the body, have been found due to experiments on the International Space Station . Protein crystals that are created under microgravity are larger and more precisely ordered than those on Earth . Research into proteins is important in medicine for the development of drugs and necessary treatments for illnesses.  Stem cells are also affected by microgravity  and investigations into this could comprehend the properties of stem cells.
\\ \\Research into the effects of microgravity is important for space travel for two different reasons. One is the effect it has on the human body and one is possible uses of microbes in space.
\\ \\Astronauts have a lowered immune system in space  so any change in microbe behaviour could affect them. To ensure the long term health and survival of space explorers, the behaviour of harmful microbes needs to be known. This is particularly relevant at the moment, with the manned Mars mission . In order for the expedition to be successful, all measures have to be taken concerning astronauts? health.
\\ \\Microbes could also be useful in space travel in the future for life support, biofuel and biomining. As they behave differently under microgravity, they cannot be utilised until their behaviour is researched and fully known or until it can mimic that on Earth. This will be discussed in full in later chapters.

\section{Effects of Microgravity on Microbes}

Studies have shown three common effects of microgravity on microbes. There is an increase in growth rate, microbes become more virile and there is an increase in biofilm production. Although it is not fully understood which mechanisms cause these, reasonable suggestions have been made. These are explored below. 
\subsection{Causes}
Convection is a form of energy transfer due to heat . It is attributed to hotter materials in a substance rising and colder materials in a substance sinking. Hotter materials are less dense and therefore lighter and colder materials are denser and therefore heavier, causing the movement. It is an effect of gravity. 
\\
\\Brownian motion is a form of diffusion . It is the random movement of particles in a fluid. It occurs where the lighter molecules of the fluid collide with the heavier particles, causing small fluctuations in movement. The fluid molecules are in random thermal motion and create an imbalance of forces on the particles. A visual representation of Brownian motion may be seen in diagram [THIS], where the large yellow particle is the particle, the smaller black particles are the molecules of fluid and the blue line is the path.
\\ \\It is believed that one of the reasons for the change in behaviour of microbes is that microbes only undergo one form of motion; Brownian motion. In space, the effects of gravity are too low for convection to happen . 
\\ \\Convection is thermodynamic and creates relatively fast movement of the particles. Convection creates sedimentation; layers of more dense particles at the bottom and less dense particles at the top of the culture . Convection leads to mixing . Both of these processes contribute to microbial behaviour as they affect how microbes receive nutrients and therefore survive. Convection is also a factor of how microbes grow and their shape.  The distribution of nutrients in a fluid is highly dependent on the convection currents and microbes will adapt their shape in order to receive the optimum amount of nutrients. Convection separates by-products from the cells themselves.  These by-products can also have an effect on the microbes? growth.
\\ \\There are many indirect reasons for an increase in growth rates as microbial growth is not a simple process but they all share one cause; the reduction of extracellular mass transfer.  This has been observed in many of the studies done. Lack of convection means that mass is only transferred due to Brownian motion, which is a lot slower and less efficient. This leads to a change in nutrient and by-product distribution which affects lag time and metabolism rates.
\\ \\The decrease in lag time and the increase in metabolism could contribute to the increase in growth rates. Lag time is a phase in the microbial growth cycle in which microbes adapt to the surrounding and there is no cell division, only cell growth.  Microbes synthesise necessary proteins, co-enzymes and vitamins for growth and repair in this time. A reduced lag time often means an increased growth time.  It is thought that the decrease in lag time in microgravity conditions is because of the lack of separation of by-products. Many of the substances microbes excrete are useful for the growth and repair in the lag phase and, in an absence of convection, stay close to the cells. It has been shown that carbon dioxide, a common by-product, reduces lag time and excreted enzymes may help with growth . The lack of layers in the microbe cultures may also be responsible for decreased lag time. Instead of the top layer receiving all of the nutrients and the bottom layer receiving all the by-products, the particles all have a relatively equal chance of receiving either. This is because there are no layers and the particles are instead positioned relatively randomly . It is believed that that the efficiency of the metabolism of the microbes increases for the same reason.
\\ \\Studies have suggested that without convection currents, cells may grow in a similar manner to protein crystals.  This is because the growth of crystals on Earth are restricted by the unstable depletion zone; an area around the microbe of low protein concentration prone to small convection currents. In microgravity, there are no convection currents so the depletion zone is quasistable which causes larger and more uniform growth of the crystals. \\ \\
Doesn?t appear virility and biofilm is understood/known at all. I don?t really understand virulence/biolfilms but I will try and explain them and relate to reasons already mentioned.

\section{A1}

\section{Microbial Fuel Cells}
There has been much discussion, ranging from wild speculation to serious scientific debate about the prospect of human survival and prosperity in environments beyond our own earth. Here, I will review some of the factors that will prove crucial to the success of future exploration and survival, in particular: the use of Microbial Fuel Cells (MFCs) as a source of both electrical energy and clean water, using wastewater as a source of energy, and creating a self-contained ecosystem capable of supporting life. 

A MFC is a device that converts chemical energy stored in the bonds of organic compounds to electrical energy via catalytic reactions by microorganisms. In a typical MFC, there is an anaerobic (free from the presence of oxygen) anodic (negative electrode) chamber and an aerobic cathodic (positive electrode) chamber between which exists a proton exchange membrane (PEM) and an external wire with a load resistor. The microbes are grown under the anaerobic conditions in the anodic chamber with the addition of an organic substrate as fuel.  The fuel is oxidised by the microbes in the anodic chamber during an Extracellular Electron Transfer (EET) process that generates protons and electrons, in addition to a by-product of carbon dioxide, and then reduced at the cathode, combining protons, electrons and oxygen to form water. \cite{du2007state,logan2006microbial}

%\begin{figure}
%\centering
%\includegraphics{images/1MFC.png}
%\caption{A diagram of a two-chamber Microbial Fuel Cell \cite{du2007state}}%\label{fig:1MFC}
%\end{figure}

Inside the anodic chamber, there are three ways in which electrons can be transferred to the anode following the oxidisation of the organic matter: by respiratory enzymes that are usually bound to the inner cell membrane such as electrochemically active cytochromes found among Geobacteraceae, by a mediator such as potassium ferric cyanide that acts as an artificial electron carrier, or by using bacteria that produce their own mediators such as Shewanella putrefaciens that produce soluble quinones.\cite{ghangrekar2006wastewater} \cite{min2004continuous}

The absence of oxygen in the anodic chamber ensures that the only acceptor of the electrons is the cathodic chamber, and thus the electrons are transported between the electrodes along the external wire, creating an electrical current. The protons pass through the proton conducting PEM to the aerobic cathodic chamber where they combine with the oxygen and electrons to form water. The key to the process is utilising the oxidation of the organic matter by the bacteria, and the reduction process through which the electrons are acquired by the oxygen molecules in the cathodic chamber. The result is the production of water, carbon dioxide and electricity. This process of oxidisation here is similar to that found in existing water treatment processes; however, bacteria that can transfer electrons to the surface of an electrode replace methanogens resulting in a process that is more environmentally healthy.\cite{ghangrekar2006microbial} There are further benefits in the use of MFCs, which include carbon neutrality (the carbon dioxide released from the biomass was originally absorbed from the environment through photosynthesis, unlike the extra release from the burning of fossil fuels, the ease in which the fuel can be obtained and replenished and the relative simplicity of the system. Most notable however is that moreorless any organic material can be inputted as fuel.  \cite{liu2004production}

It is clear then that this kind of technology has applications in space exploration. An energy source that uses unwanted human waste and produces pure water could potentially tackle two of the most pressing needs for human survival. There are already plans afoot to utilise MFCs for powering planetary robotics such as ?micro-rovers?. The US Naval Research Lab were able to produce an appliance weighing approximately 1kg with its electronics and controls powered by some of the energy from the MFC, with surplus energy being diverted into a capacitor/battery until sufficient energy was stored to power a ?tumbling? locomotion system. In this instance, Geobacter sulfurreducens, an anaerobic, long-life bacterium was used. The advantage of this system is that it will continue to generate power, and recharge the on-board capacitors for a duration suitable for long exploration missions. \cite{society2012microbial}

To better understand better MFCs can benefit human exploration missions, it is worthwhile considering how different configurations alter the power output, economy and practicality. Firstly, the use of a mediator to transfer electrons to the anode, substances that are suitable for this task include potassium ferric cyanide, neutral red, methyl viologen and thionine. While the high concentration of electron carriers can lead to higher power output, these substances can be expensive to source, and are often toxic and harmful to microorganisms.\cite{logan2006microbial} Therefore, when seeking long-term, self-sustaining power supply, it is more practical to utilise bacteria that use respiratory enzymes, or produce their own mediators.  \cite{min2004continuous} \cite{ghangrekar2006microbial}

There is also evidence that a mixed culture of microorganisms (even those that cannot transfer electrons to the anode directly) can lead to a six-fold increase in the current output, however I will not consider the exact composition of bacteria required here.\cite{ghangrekar2006microbial}

Another consideration of the system configuration is the material used for the PEM. The requirements are that protons are able to diffuse, while the oxygen, bacteria and fuel transfer between the two chambers is minimised. Oxygen entering the anodic chamber can have a large, negative effect on the coulombic efficiency (electrons collected/maximum possible collection) \cite{logan2006microbial}.  Designs as simple as porous ceramic plates and salt bridges have been used for this purpose in the laboratory \cite{min2005electricity}, although the very presence of a membrane presents an issue in the treatment of wastewater, as any such membrane would be subject to fouling, particularly with the presence of suspended solids and soluble contaminants. Monitoring the fuel cells for when this occurs and the associated maintenance cost of supplying and replacing the membrane presents another problem when seeking a system that requires long-term stability. 

A solution to this problem could be to use a membrane-free cell. In this setup, wastewater flows through both the anaerobic and aerobic sections, with the protons conducted by the water. It has been shown that this method can still treat wastewater containing organic contaminants. A PEM is likely to be a better conductor of protons than water, however it may actually add to the overall internal resistance of the system. \cite{ghangrekar2006wastewater,logan2005simultaneous}

\begin{figure}
\centering
\includegraphics[scale=3.0]{images/2membranelessmfc.png}
\caption{\cite{ghangrekar2007performance}}
\label{fig:2membranelessmfc}
\end{figure}

There is a significant reason why this method is the most applicable for use in space exploration; there is no mixing required other than the flow of liquid into the system, which can be achieved using a pump system. This is an integral factor for the functionality of the device in a micro-gravity environment, and it is unique to this particular setup. \cite{min2004continuous}

With a configuration that could plausibly operate in extra-terrestrial conditions, it is necessary to consider ways of maximising the power output. Firstly, like in all systems, electricity will only be generated if it is thermodynamically favourable.  The maximum energy output can be calculated from the Gibbs free energy:

$\Delta G = Delta G_0 + RT ln(II)$

$G$ is the change in Gibbs free energy, $G_{0}$ is the Gibbs free energy under standard conditions (1 bar pressure, 298.15K), R is the universal gas constant, T is the temperature and II is a unitless reaction quotient. The simplicity of MFC design allows the electromotive force to be calculated from the potential difference between the two electrodes (electromotive force of the system): 

\begin{equation}
W=Q E_emf = -G
\end{equation}


Q = Ne, the number of electrons transferred to the anode, multiplied by the charge of an electron. The two equations can be combined to give the EMF in terms of the Gibbs free energy:

\begin{equation}
E_emf = - G_0 / Q
\end{equation}

And at standard conditions, we take the reaction quotient II to equal 1, obtaining a standard cell emf:

\begin{equation}
E^{0}_{emf} = - \frac{G_0}{ Q}
\end{equation}


The reaction can now be expressed in terms of potentials:

\begin{equation}
E_emf = E^{0}_{emf} - RT/Q ln (II)
\end{equation}


This calculated EMF here provides the theoretical upper limit for the voltage and thus the current and power. The actual values of potential at the electrodes can be analysed in terms of the reactions occuring. An example anodic reaction would be the oxidisation of acetate that is represented by:

\begin{equation}
2HCO_{3}^{-} + 9H^{+} + 8e^{-} -> CH_{3}COO^{-} + 4H_{2}O
\end{equation}


The potentials can then be compared to the normal hydrogen electrode that has zero potential under the standard conditions stated above. The actual potential can then be deduced by assuming the activities of different species is equal to their concentration, so in this case where the reaction is acetate oxidation:

\begin{equation}
E_{An} = E_{An}^{0}  - RT / 8F ln ([CH_{3}COO^{-}]/[HCO_{3}^{-}]^{2} [H^{+}]^{9{}})
\end{equation}

F is Faraday's constant.

And at the cathode, where oxygen accepts the electrons in the reduction process:

\begin{equation}
O_{2} + 4H^{+} + 4e^{-} -> 2 H_{2} O
\end{equation}


\begin{equation}
(E_{Cat} = E_{Cat} ^ {0} - RT/4F ln(1/ p O_{2} [H^{+}]^{4}
\end{equation}


The pH of the cathode solution can affect the overall cathode potential, but the cell emf can now be expressed in terms of the two potentials.\cite{logan2006microbial}

\begin{equation}
E_{emf} = E_{Cat} - E_{An}
\end{equation}
 
Of course, there are many other factors that reduce the output from the theoretical limit, such as: resistance in the cell and wires (ohmic losses), energy lossed during the oxidisation and reduction processes (activation losses), energy gained by the bacteria transporting electrons across a potential difference to the anode (bacterial metabolic losses) and loss through mass transfer at high current densities (concentration losses).\cite{logan2006microbial}

Concentration losses are a particularly concern when assessing the plausibility of power generation in space. With a high oxidisation rate, a large number of electrons diffuse towards the electrode but this process of mass transfer can be slow and limited. This can lead to a limit in both the discharge of the oxidised species from the anode and collection of reduced species at the cathode. This can cause an increase in the anode potential or a drop in the cathode potential, and reduce the overall output of the system. This scenario is likely to be more prominent in a poorly mixed system, as is expected in a micro-gravity environment, where diffusional gradients may inhibit transportation and increase concentration losses. The further use of pumps could increase the amount of mixing, but it is important to ensure that the power output exceeds the amount that the pump consumes. Similarly, the use of the centrifugal force as an artificial source of gravity may circumvent this problem. Other innovative methods of achieving mixing in micro-gravity are discussed later in this report. 

It is noteworthy that the maximum theoretical value of potential has dependence on temperature. This is due to both thermodynamic limits, and the activity of microbes in different and varying environments. Assuming that the system is set up to ensure that the reaction will always be thermodynamically favourable, the effects of temperature on the microbes has a more significant effect on the overall performance of the system. L.H. Li et al \cite{li2013effect} discovered that electricity generation from a MFC fuelled by sludge from a river occurred between 283-316K, with no activity recorded outwith these values. This affect can be attributed to both the decrease in microbial activity, but also an increase in the total internal resistance. The highest power output was achieved with the device operating at 306K, suggesting that successful operation of an MFC may require a controlled environment to ensure sustainable power output.

\begin{figure}[ht]
\centering
\includegraphics{images/3effectoftemperature.png}
\caption[\cite{li2013effect}]{A graph showing the effect of temperature on internal resistance and power density. The variables were found to be closely related, with optimal performance occuring at around 37 degrees celsius.}
\label{fig:3effectoftemperature}
\end{figure}

Using the thermodynamical approach described previously, Logan et al \cite{logan2006microbial} reported a theoretical maximum potential difference of 1.1V. This reflects ideal performance, but is not a particularly reliable figure due to the complex respiratory chain through which the oxidisation process occurs, a process that varies from microbe to microbe and depends on the external conditions. The actual obtainable value is more likely to be between 0.5-0.8V, similar to that of a hydrogen fuel cell.\cite{du2007state,logan2005simultaneous}

If the potential difference between the two electrodes is known, it is simple to calculate the power of the system using Ohm?s Law:

$P = E_{emf} ^{2} / R_{ext}$

Where $R_{ext}$ is the external resistance, usually in the form of a load resistor. A calculation more indicative of the performance is that of the power density, the power output per unit anode surface area.

$P = E_{emf} ^{2} / R_{ext} A_(An)$

A measurement of the power by a mediator-less MFC, continuously fed wastewater as fuel yielded a steady power output of 6.73mW/m2 \cite{ghangrekar2006microbial}, but elsewhere an isolated maximum value of 1.5W/m2 has been reported \cite{logan2008microbial}. There is a large disrepency in the figures reported, however taking into account the extra concentration losses expected in a micro-gravity environment, the power output achievable at a sustained level at present would only be sufficient for low powered appliances, or for charging capacitors for long periods to enable short periods of high-powered discharge. Connecting a number of MFC devices in a series could increase the output, although it is unlikely that MFCs alone would provide a single, sustainable source of electrical energy.

While more research needs to be carried out to find the ideal system design to maximise power density and output, it is worth examining the merits of this system as an appliance for wastewater treatment alone. The treatment of wastewater involves the removal of organic matter in the form of Chemical Oxygen Demand (COD) and Biochemical Oxygen Demand (BOD). COD is a measure of the organic matter (that cannot be oxidised biologically) present in water, measured in mg/L. It is a good indicator of water quality. BOD is the amount of dissolved oxygen needed by aerobic biological organisms to break down organic matter in water. \cite{surez2005determination} 

Existing methods of wastewater treatment require large amounts of energy and economic expense In many existing processes, energy is recovered as methane, but it is not very useful, and often flared \cite{ghangrekar2006wastewater,chang2005improvement}. COD removal efficiency is the most common measure for water treatment efficiency. Using the membrane-less device described earlier, operating over a period of 78 days, [ghangrekar et al] observed a steady increase in both the removal efficiency over time before reaching a maximum value of 90.86% with 50? applied external resistance:

\begin{figure}[ht]
\centering
\includegraphics{images/4CODremoval}
\caption[\cite{ghangrekar2006wastewater} A table displaying BOD and COD removal under varying conditions]{}
\label{fig:4CODremoval}
\end{figure}

As a useful comparison, a study of the COD and BOD removal efficiency of two urban waterwater treatment plants in Bangalore, India revealed BOD removal efficiency of 94.98\% and 97.60\% respectively, and COD removal efficiency of 76.26\% and 91.60\% respectively \cite{pravikumar2010assessment}. Already, the results from the membrane-less MFC are comparable to existing water treatment data. There is further evidence that MFCs have the potential for widespread use in wastewater treatment \cite{fornero2010electric}, and with further refinement, particularly in finding a diverse, mixed-culture of microbes capable of oxidising a wide variety of organic material and suitable scaling up of the MFC design to handle large volumes of continuously inputted wastewater, there is scope for the use of these devices on future exploration missions. Having shown that the devices are suitable for wastewater treatment alone, the additional electricity generation capabilities can relieve the total energy demand by contributing energy to battery/capacitor powered devices. The underlying simplicity and regenerative nature of the device leaves the door open for further development and modifications to perhaps increase the power density in future.

\section{MELiSSA}

While much of the discussion so far has considered future uses and adaptations of existing technology for use in micro-gravity environments, there are some microbiological technologies already in development for this very purpose. A prime example of a project championing such progress is MELiSSA (Micro-Ecological Life Support System Alternative), a project led by the European Space Agency (ESA) that combines a vast number of research departments, corporations and technologies from around Europe aiming to develop a sustainable, regenerative life supporting ecosystem system that could prolong survival capabilities on lunar exploration missions, or further missions to Mars and beyond. \cite{agency2008melissa} 

ESA estimate that a mission to Mars would require 30 tonnes of supplies. This, clearly, is unattainable. As such, MELiSSA aims to develop an eco-system that is entirely self-sustaining with efficiency as close to 100\% as possible, recovering water, carbon dioxide and food from human waste, carbon dioxide and minerals. The project aims to support a human crew by 2020-2025. An important component to the project is the recognition that human waste may become one of the most valuable resources in the success of human space exploration. One example of a technology that has been developed successfully is known as Biostyr (Veolia Water Solutions and Technologies \cite{veolia}); like an MFC, it is a technology that can treat wastewater and remove nitrogen and carbon content from water in one cell, using bacteria to oxidise the waste as fuel and recover clean water. While the technology was developed as part of project MELiSSA, it is already being used in the treatment of more than 100 billion cubic litres of water per day and could play a vital role as part of a self-sustaining ecosystem. Further testing of others such wastewater treatment is taking place at the ESA?s Concordia Station in Antarctica, one of the most inhospitable areas on earth that is inaccessible to supplies and communications for 9 months a year.  The Grey Water Treatment Unit used at the site recycles approximately 85\% of the facilities water. \cite{salem2009coldest}

Another development being pursued as part of this project making use of microbes, is the development of an air revitalisation system based on the BIORAT experiment \cite{agency2000exemple}. There are two compartments to this system, one is the consumer compartment (CC), where an oxygen consuming organism is habited (humans, or for the purpose of this experiment, mice), the second compartment contains a photobioreactor (PBR) with a photosynthetic (carbon dioxide consuming and oxygen producing) organism, such as Spirulina platensis. The air is circulated with a pump, and is enriched with carbon dioxide in the CC due to the metabolic activity of the mice, this carbon dioxide is then consumed in the PBR and the air is enriched with oxygen again. The experiment intended to show that an artificial ecosystem can be created first under earth conditions, then under microgravity conditions on the International Space Station (ISS).The intention was to control the oxygen supply based on the demand and the activity of the mice by modifying the light intensity incident on the photoreactor. 

\begin{figure}[ht]
\centering
\includegraphics[scale=1.2]{images/5photobioreactor}
\caption[\cite{agency2000exemple} A diagram showing the basic functionality of the experiment.]{}
\label{fig:5photobioreactor}
\end{figure}

The setup is extremely simple, with oxygen flowing from the PBR to the CC, being consumed by the mice and returned to the photobioreactor as carbon dioxide. The light intensity incident on the PBR determines the level of photosynthesis and thus the amount of oxygen produced. 
The three most important variables for the sustainability of this system are the speed of recovery of oxygen from the photobioreactor, the removal of carbon dioxide from the consumer compartment and the software control of the light intensity. Further problems can arise through the coupling of pressure and air flow. The pressure in the CC needs to be maintained without inhibiting the inflow of oxygen or the outflow of carbon dioxide. Furthermore, air passing through the photoreactor can collect moisture and add unwanted condensation. It is important that these conditions are monitored, to ensure a quick response to any unexpected variable change. 

The final problem that is fundamental to use of microbes in space is how the system will perform in microgravity. The absence of gravity prevents the natural separation of gas and liquid phases, which are necessary in the photoreactor. Injecting gas through liquid in the photoreactor will not provide sufficient separation. The solution proposed was to create artificial gravity (centrifugal force) by rotating the PBR that would allow the separation of the liquid and gas phases, ensuring that the oxygen flow was not restricted. 

\begin{figure}[ht]
\centering
\includegraphics{images/6oxygenlevels}
\caption [\cite{agency2000exemple}]{}
\label{fig:6oxygenlevels}
\end{figure}

The x-axis is the time in days, the y-axis (right, grey) is the intensity of the light incident on the PBR and the y-axis (right, black) is the oxygen level. 
The results provided a strong demonstration that this system is sustainable over a long period. For 21 days, the oxygen supply was maintained at nearly constant level of 20\% (right axis), perfectly sufficient for human consumption. It was also found that the carbon dioxide levels stabilised at 0.15\%, the pressure at 1Atm and the moisture content at 55\%. The successful demonstration of the technology over a long time period offers great promise using microbes in an air revitalising system.  

\section{Edvinas Bit}

Space exploration is inevitable by technologically advanced society. It is not only because of curiosity of humans and the desire to expand boundaries of our current knowledge, but maybe even more important is the fact that our world is finite in space and resources. In today’s industrialised world these resources are consumed very rapidly and it is only the matter of time when the scarcity of natural resources on Earth will be felt. Even though a lot of materials and
metals can be recycled, the large amount of power is used to do that and radioactive elements such as uranium which is used in nuclear power plants are entirely consumed without being renewed ever again. If we consider hydrogen fuel cells for electricity generation which are far more efficient and environmentally friendly than traditional means of power generation the electrodes of it are produced from platinum group metals such as platinum, iridium, osmium, rhodium, ruthenium and palladium. And the high cost of these types of fuel cells is entirely because of extreme rarity of platinum group metals in the Earth’s crust. The global average concentration of these metals is only 4g per ton of ore. Also automotive industry is the largest
user of platinum and palladium catalytic converters. Another metal group is rare Earth's metals which are used for magnets, x-rays and MRI scanning systems, TV screens, other electronic devices and for renewable power generators. Also mining these metals destroys environment and natural habitats for other animals on Earth. It is well known that wars are usually fought because of the land and resources that it holds. On the other hand, space holds vast quantities of
resources which seems infinite compared to just one planet. Inner solar system planets, their moons and asteroids can be exploited.

However, resource gathering industries in space would just be the pioneers of all the new industries and businesses, as power plants, research laboratories, habitats for humans, hotels or entertainment parks would also appear. The establishment of such a large scale projects would require huge amounts of material to be exported from Earth to space. However, there are two significant factors inhibiting the establishment of these kinds of projects. First, the cost of launching anything from the surface of the Earth to orbital space remains very high. For example, current launching systems have a cost of \$ 2,000.00 per kilogram of payload for low Earth orbit and from \$16,000.00 to \$50,000.00 for geostationary orbit. Another aspect is that soft landing to other planets and moons and lifting from them requires a lot of energy input. This all leads to the necessity of having all the materials which are required for building and life sustainability in space reachable by hand. This means that there is economical and practical necessity to produce everything on space instead of constantly sending supplies from Earth.


Thanks to the increasing technological advances in biology and biotechnology almost all of the difficulties mentioned can be overcome using microbes. Proteomics, DNA sequencing and genetic engineering opened the doors to vast amounts of applications of microorganisms. As it is possible to include genes or deactivate them inside bacteria genome the new types of bacteria
can be engineered for the purpose of doing some specific task. This is the major benefit for using bacteria. Other thing is that the cost of sustaining them is low and they reproduce at highrates. This section will explore potential uses of microorganisms in space through the lenses of current advances in biotechnology on Earth.

\subsection{Microbial survival in outer space environment}

Environment in space is unimaginably different from the environment here on Earth. To use microbes in space they have to survive in these extreme environments. Depending on their use they need to survive drastic pressures and temperatures, ultraviolet and ionizing radiations,
different ranges of pH and effects of microgravity. Fortunately, to this date there are a lot of extremophiles investigated which can be found on Earth. They manage to live even in the harshest environments. Also, they are present in different domains of life, such as archaea, bacteria and eukarya.

In 2009 EuroGeoMars campaign was organised to perform multidisciplinary astrobiology research at Mars Desert Research Station (MDRS). This station is situated in the desert of southeast Utah.\cite{} It has very dry soils similar to Mars and they are subject to wind erosion. The daily variation of temperature can be around 20 o C with temperature differences of -36 to 46 o C.
At MDRS variety of clay minerals has been detected. Also it has red-coloured hills, soils and sandstone due to the iron-oxide present which is the same mineral that gives red colour to Mars environment. Research concluded that all domains of life were detected and with high occurrence and diversity over short distances. This means that these potential bacteria could be introduced in Mars environment and Mars soil could provide minerals for the growth of these microbes. [1, 2].Another research made by L.C. Kelly, Charles S. Cockell and Stephen Summers show how non-spore forming microbes such as Proteobacteria(Pseudomonas) can survive very high pH levels which occurs due to the high concentration of ammonia(NH3) which is normally highly toxic to organisms. The experiment shows that these microorganisms can survive temperatures as low as -80 o C which is the temperature of Mars surface at some
points.\cite{}[3].


\subsection{Biomining}
\todo[inline]{think of another way to start this sentence}
As previously mentioned mining outside Earth would be the start for developing habitable systems in space because there would be no need to bring materials from our planet, all of them would be mined and manufactured on spot.

There is a clear advantage of exploiting the moon first. The short distance from Earth and the Moon and the fact that it is orbiting the Earth rather than the Sun or any other planet makes it perfect candidate to start implementing technologies for initial tests.\cite{} Also, quite a substantial amount of water deposits discovered in 1990s makes the Moon potentially a good place for
permanent habitats. As moon was already visited by humans, there is large amount of information about the geology and its mineral composition. Analysis on the samples gathered by Apollo missions reveals that the lunar soil contains aluminium, silicon, iron, calcium, magnesium, oxygen and titanium in various compounds.\cite{} Also, very valuable Helium-3 which is used for various medical and nuclear applications exist on Moon. Helium-3 is extremely rare on
Earth, because it is lost through dissipation by upper reaches of atmosphere. However, it is abundant on the Moon due to solar winds.

Other attractive candidates for resources are asteroids due to their low escape velocities and richness of minerals including carbon, sulphur, phosphorus, water and platinum group metals which are in higher concentrations than even the best mines on Earth. Table 1. bellow shows materials found on asteroids in space.

\todo[inline]{insert table}

Biomining is an alternative to traditional physical and chemical methods of mineral extraction. There are many advantages of using this method instead of any other. Microbes are used to extract elements from low-grade deposits in which traditional means of mining would be very inefficient on Earth. Biomining does not need vast amounts of energy for roasting and smelting of ores and chemicals such as cyanide which is used to leach elements from rock, especially
gold. Microbial mining plants are simple to operate and the construction and expansion of these plants are quick even in the remote locations where mineral deposits are much more difficult to exploit. The prototype plant created by BioCop TM was built in Chuquicamata and ready to use just in four weeks. The plant does not need much steel and other building materials and also
human labour for operation. When considering biomining in space these things are especially valuable. The main strategy of metal recovery is called bioleaching. A lot of sulphides of metals are insoluble in water such as zinc, copper, nickel and cobalt. However, sulphates of these metals are soluble and can be extracted. The main role of microorganisms is to generate ferric iron and acid. Iron $\mathrm{Fe}^{3+}$ ions are used to oxidize the ore. These ions are regenerated from $\mathrm{Fe}^{2+}$
ions, for example in pyrite $\mathrm{FeS}_{2}$. Oxygen is used as oxidiser by bacteria. The remaining products are soluble and can be recovered by other means. Most important organisms are iron and sulphur oxidizing chemolithrops. They grow by autotrophic means by fixing $\mathrm{CO}_2$ from atmosphere. These organisms obtain energy not by using sunlight as most organisms but by using ferrous iron or reduced inorganic sulphur compounds as an electron donor and oxygen as electron acceptor. These organisms can withstand very low pH levels as sulphur acid is produced during oxidation process. Usually mineral oxidation processes using microbes operates in pH as low as 1.4.\cite{} Some organisms can even use ferric iron in place of oxygen as electron acceptor. This is important in places of bio-reactor where oxygen is low and cannot be supplied. Most important organisms in these reactions which operate at lower temperatures then 40$\,^{\circ}\mathrm{C}$ are iron- and sulphur- oxidizing Acidithiobacillus ferrooxidans There is evidence that at higher temperatures bioleaching processes occur faster.\cite{}

\todo[inline]{insert}

Above is bioleaching reaction with chalcopyrite in order to extract copper. First step occurs spontaneously in water. Then bacteria oxidises $\mathrm{Fe}^{2+}$ by taking its electron and giving it to
oxygen. This creates active $\mathrm{Fe}^{3+}$. In third step bacteria oxidises sulphur and sulphur acid is created which in turn creates low pH. At low pH active iron reacts with primary chalcopyrite and as can be seen in step four Copper and Iron elements are recovered in solution as well as sulphur.


There are several reactor technologies for biomining but the most useful for application in space seems to be heap-reactor technology. Heaps are constructed to already predetermined dimensions using graded ores. In this type of reactors agglomerated ore is piled on impermeable base and Leaching solution which is highly acidic is permeated through the crushed ore. Then it is supplied with an efficient leach liquor distribution and collection system. Heaps are irrigated
from above with these acidic liquors and aerated from bellow to provide carbon dioxide required by autotrophic microorganisms and oxygen to promote sulphur and iron oxidation. Microbes can grow in the heap and create ferric iron and acid which in turn engages dissolution of mineral.\cite{} Finally leach solution that drain from the heap is sent for metal recovery. Advantages of these reactors are that they are rapid to build and to operate, they require low capital cost and all solutions are recycled. The systems are closed and don’t make contact with
outer atmosphere. Disadvantages are that often environment in the systems is not homogeneous which requires greater balancing.

\todo

There are other interesting microorganisms which can be applied for mining. Rakshak Kumar at al. show the finding of bacteria in largest sandstone-type uranium ore deposit in India.\cite{} The diverse group of bacteria which are able to tolerate substantial concentration of uranium and other heavy metals were found. The analysis of bacteria reveals the ability of bio-sorption of
metal cations via complexation with cellular ligands, they can oxidise uranium, also bio-mineralise metals. Other two papers by Abhilash at Al show the performance of bacteria in uranium bioleaching processes. As much as 98\% of uranium was bio-recovered. This technology can be applied in space with potential need of uranium. Also the bacteria can be used in nuclear waste-sites or on Earth or even in future space exploration to collect and clean
uranium from potential accidents as some robots use mini nuclear reactors for power generation, for example Curiosity rover. More-over, Gotz haferburg at al. investigated polluted environments at a former uranium mining site in Germany. One of the samples showed promising results, the capacity to uptake heavy rare Earth elements which are very toxic to micro-organisms

Recently interesting review by Debaraj Mishra and Yound Ha Rhee was written investigating microbial leaching of metals from solid industrial waste. Major solid wastes, electronic components, batteries petrochemical catalysts, fly ash and other things were subjected to microbial leaching.


ith addition of iron and sulphur to coal fly ash a lot of metallic elements were recovered, such as aluminium, nickel, zinc, copper, cadmium and chromium. In technological age new electronic devices are created constantly and recovery of harmful metals from electronic waste has become an important research goal. Thermophilic and mesophilic(grows in moderate temperatures) bacteria were used to bioleach electronic scrap and were able to recover more than 90\% of Ni, Cu, Zn and Al. Also the mixed consortium of microbes and fungi have-been
termed as computer-munching microbes. This consortium was able to dissolve heavy metals as Cu, Ni and Zn in single step or two step processes. Also, they showed promising results for copper extraction from circuit boards. Another study by Mishra et al.\cite{} was made to investigate bioleaching process in another major solid waste- Li-Ion and Ni-Cd batteries which are used in
almost all electronic devices. In Ni-Cd batteries 100\% cadmium was recovered and 99\% of Cobalt in Li-ion batteries. In the study by Gerayeli at al. bioleaching was applied to petroleum catalyst and 35\% of aluminium, 83\% of molybdenum and 69\% of nickel were recovered using acidophilic Archaea Acidianus brierleyi. 

\todo[inline]{try to start this paragraph using less colloquial language}
Looking at the research done it seems that bacteria is really good candidate for hard waste recycling. This developing technology would find the use in space where small amounts of materials from batteries and electronics could be rapidly recycled and used again. Also, as so much research is done in geomicrobiology for biomining purposes and quite a lot of elements can already be extracted, it seems quite reasonable to think about near-future possibility of mining in space.

\subsection{Conversion of plants to biofuels}

This is an exciting are of research for potential application to space exploration. Keeping in mind the possibility that humans could create habitats on Mars in which plants will be grown, they would find the oil like fuel useful for different kind of tasks. Oil is abundant on Earth. However, it has not been found anywhere outside the Earth to date, as the oil is created by pressurising living matter throughout millions of years. However, very recent studies of Mirko Basen et al., (2014) \todo[inline]{do not use harvard referencing} and Somiyo Kanafusa- Shinkai at al., were made investigating thermophilic
bacterium Caldicellulosiruptor bescii ability to degrade unpretreated cellulose(The walls of plant cells).

\todo[inline]{think of another way to start this paragraph}
Basically during evolution plants evolved in a way to guard themselves from different
microbial and enzymatic attacks. The major immune components to degradation of plants are
the glycine polymers cellulose, hemicellulose and the polyaromatic lignin. The problem is that
quite substantial thermochemical and physical pre-treatments are required to enable cellulose to
be enzymatically hydrolysed and to date no organism is known that at the same time can
ferment polysaccharides(polymeric carbohydrates, sugars) from plant biomass and produce
biofuel at efficient rate. However, these new studies shown that this new anaerobic, non-spore
forming and extremely thermophilic bacteria is capable of degrading high concentrations of
unpretreated switch grass, of up to 200 g/L which is already relevant to industrial conversion. It
is the only microbe which is capable of doing this. However, this research is only the start and much more work will be needed to make this bacterium exploited. By degrading cellulose and
other components waste is used to produce fuels and other organic chemicals. This would be a
great renewable fuel from plants which would be easy to make.

\subsection{Separation}
Conservation and recycling of materials is vital in any type of long duration space exploration due to limited available volume, the need to minimise mass and because power is a limited commodity. 

One area of interest is how to separate microbes from mixtures in microgravity. This would allow the microbes to be reused and any other products to be treated suitable. This would mean that interplanetary ships could be self-sufficient for the maximum possible length of time.

Due to the lack of gravitational forces, mixtures in microgravity do not sediment as on Earth. As seen in the first chapter of this report, the continuous Brownian motion of particles and lack of convection currents means that research is required into different means of particle separation that would not be as effective or commonly used in a terrestrial environment. Due to the lack of gravity, it is found that gas and liquid do not separate from each other naturally. Instead, the gas created from reactions of chemicals or processes of microbes will form bubbles within the liquid which do not separate themselves due to buoyancy, and therefore gravity. Thus a system needs to be implemented in order to separate gas and liquid. Means of separation of different phases of matter in microgravity will now be explored.

\subsection{Microbial-fuel cells}

In recent years, major research has been focused on Microbial fuel cells (MFC). It is easy to see why this is the case as MFC is the source of renewable energy which can be used anywhere, especially as microbes are so diverse and can be sustained in various environments. The discovery of bacteria Shewanella oneidensis accelerated the advance of MFC. The advantage of
this bacterium is that it can produce bacterial nanowires under anaerobic conditions which is the main reason in advancing these fuel cells. These nanowires can facilitate direct transfer of
electrons to anodes. The basic principle of electricity creation in this fuel cell is that anode and cathode are separated by an ion exchange membrane, the solution consists of organic matter which is used as a fuel to microbes. The property of bacteria to oxidise is used to do the work.The basic scheme drawn by Deeksha Lal is shown in 

\todo[inline]{\cref{fig:}.}
Even more interesting is that the assortment of organic fuel used in the batteries is comparatively large.
One of the possible applications is waste water treatment. This is especially important in space where all the organic waste could be converted into energy at the same time removing unwanted stuff. Also, it was found that diverse microbial communities are more stable and robust due to the nutrient adaptability and resistance to stress. More-over, having wide variety of organisms widens range of fuels that can be used for oxidation.

Nabuo Kaku at al. presented quite an interesting idea of application of MFC system to electricity generation in rice paddy field. This is a flooded parcel of arable land used for growing rice and other semi aquatic crops. When a paddy field is flooded, the soil immediately below the surface of the field becomes anaerobic which is a good environment for bacteria used in MFC. It is known that potential gradient is formed between the soil and the flooded water.
The group installed electricity generating system in a rice paddy field during rice-cropping season and examined how much electricity was generated. The system they installed utilizes the natural potential gradient between the sediment and upper water. Electrons are then released by the microbial oxidation of organic matter flow from anode (in sediment) to the cathode (in water) through an external circuit. Up to 6mW/m 2 were observed by this group and up to
20mW/m 2 by other groups. The output is not large but electricity flow was constant and could be improved in the future. This is quite a novel approach for combining constant and passive electricity generation without requirement for human labour and at the same time growing rice which is one of the most protein rich and nutritious crop. Figure of the set-up of paddy-field generating electricity is below.

\subsection{Biofilm use and soil formation}
One of the most interesting properties of microorganisms is that they can form biofilms. These
biological films are complexes of microorganisms which stick together by their outer
membranes on some surfaces. In fact, biofilms predominantly form in most natural settings and
a lot of times different type of bacteria are found within them. Biofilms can form on pipelines,
catheter, teeth, plant roots and even on the lungs of cystic fibrosis patients. The formation stars
to develop when single celled organism switch from nomadic lifestyle to multicellular, where
structured communities and cellular differentiation emerges. Basically, there are two general
types of biofilm formation. This is due to the motility of microorganisms. Motile unicellular
organism can use energy and actively move around, for example using flagella (tale on the side
of bacteria). In the case of non-motile organisms individual bacteria start to increase the
expression of adhesins outside the outer membrane. These adhesins increase the stickiness on
the surface of bacteria and this promotes cell-cell adherence. For example, in some strains of
staphylococcal species proteins expressed to surface promote cell-cell interaction and contribute
to the extracellular matrix. In the case of motile species, bacteria change their life style
dramatically. When the right conditions appear they settle down to the surface, start to produce
elements to extracellular matrix and loose the main ability to move around with flagella. There
are five basic stages in the process of biofilm formation. First is initial surface attachment, then
monolayer forms from bacteria on that surface followed by migration of other bacteria to
produce multi-layered micro-colonies. Then the stage follows where production of extracellular
matrix is highest and finally biofilm matures with characteristic three dimensional architecture.
\todo[inline] {fig: "shows very basic formation of non-motile and motile biofilm. Non motile is on
the left and motile is on the right"}.

One useful application of biofilm formation in outer space would be to control dust particles in
Mars or Moon or other planets. One of the properties of Mars is that mainly all the surface is
covered with fine dust which can damage different kinds of equipment and respiratory systems
like human lungs. Also, huge dust storms happen constantly. During the experiments made by J.
N Latch et al., Hawaiian volcanic ash which is Martian dust simulant induced real damage to
human alveolar cells. Chunxiang Hu et al. used several species of bacteria to reduce wind
erosion in sands. Tests with wind tunnel showed decreased erosion rates. Also, study by Y.
Liu, C. S. Cockell et al. investigated the ability to form deserts crusts artificially. It was shown
that just after 15 days period, crusts were formed which were able to resist the wind. One of the
reasons is biofilm formation around dust particles, which holds them together.


Microorganisms are the main reason for soil formation. Technically speaking, soil is the result
of complex biological, chemical and physical processes. However, microbes are initializers and
the main drivers of transformation and development of soil increasing its carbon and nitrogen
and other nutrient pools, which leads to the establishment of plant communities. In principal
there are two main functions of microorganisms. First is initial weathering of the bedrock
material and second is forming interfaces for nutrient turnover at vegetation free sites. Frey et
al. showed that granitic sand extracted from Damma glacier was effectively dissolved and the
main reason for this was formation of biofilms on the mineral surface. Biofilms were secreting
organic oxalic acid which decreased pH and promoted dissolution of the rock. The reason that
this topic may interest space exploration is the fact that microbes and finally plants could be
introduced into planets without any trace of life. For example, A. Bauermeister et al. were
investigating the potential growth of acidophilic iron-sulphur bacteria growth on resources that
are expected to be on Mars. They showed that these bacteria can grow entirely on Martian
syntetic regolith mixture with no added nutrients. With introduced Martian atmosphere bacteria
survived for a week by forming dried biofilms. Actually it was shown that low Mars
temperature and low oxygen pressure was favourable to survival. Biofilms are also used in
MFCs and Biofilm membrane bioreactors. These bioreactors are used for waste water treatment.
They effectively remove organic wastes, gases and elements like phosphorus and nitrogen from
the water. One of the advantages is that reactors can be quite compact and could be used for
water filtering during space flights.

\subsection{Exploitation of extremophiles in medicine}

As mentioned at the beginning of the section, extremophiles are microbes which can withstand
very harsh and changing environments. One reason is that proteins of these bacteria have very
high conformational stability. There are already developed applications of extremophiles which
are used as cell protectants in skin-care and free-protein stabilizers and also as protectants from
environment. The stability against heat denaturation of protected proteins is increased. Also,
there is evidence that extracts from extremophiles can protect and stabilize DNA. It actually
lowers the melting temperature of double stranded DNA and increases thermal stability overall.
Louis et al. showed that extracts of extremophiles stabilize cells and bacteria, such as E.coli
during drying. There are even more interesting uses. The effect of ectoine(extract from
extremophile) was tested on membranes of animal red blood cells containing haemoglobin in
vitro. The result was surprising as this element protected membrane of the red blood cells from detergents even better than some well-known stabilizers. Also, there were studies which tested
ectoine effect on skin cells against UV radiation. At the end after applying radiation doses there
was no damage to the skin. The same study also showed that ectoine protected mitochondrial
DNA of human cell culture. Although the studies were done in vitro(in glass tube, outside the
body) these findings reveals great potential and motivation for further research of extremophile
application for human protection against radiation damage and extreme temperatures. This
would be really useful for space travellers or even cosmonauts in the space stations because
today they still have to worry about prolonged exposure to space environment and damaging
radiation to their cells.

\todo[inline]{table}


\section{The Separation and Mixing of Microbes in a Microgravity Environment}
The large array of possible uses for microbes in microgravity environments has been explored in the previous sections. Thus, it is evident that researching what technology is required to make these applications a reality is needed. 


Due to the less than intuitive nature of the physics involved with microgravity this section is broken into two sections, each one demonstrating possible methods to perform simple operations on particulates in a microgravity environment. The first section will explore methods of separation and fractionation of particulates, this being a basic operation with many uses on an interplanetary space ship. The second section will look at a possible low energy method of mixing.

\subsection{Separation}

Conservation and recycling of materials is vital in any type of long duration space exploration due to limited available volume, need to minimise mass and because power is a limited commodity. One area of interest is how to separate microbes from mixtures in microgravity. This would allow the microbes to be reused and any other products to be treated suitably so that interplanetary ships could be self-sufficient for the maximum possible length of time.

Due to the lack of gravitational forces, mixtures in microgravity do not sediment as on earth. As seen in the first chapter of this report, the continuous Brownian motion of particles and lack of convection currents causes researchers to have to explore means of particle separation that would not be as effective or commonly used in a terrestrial environment. Again due to the lack of gravity, it is found that gas and liquid do not separate from each other naturally. Instead, the gas created from reactions of chemicals or processes of microbes will form bubbles within the liquid which do not separate themselves due to buoyancy, and therefore gravity. Thus a system needs to be implemented in order to separate gas and liquid. Means of separation of different phases of matter in microgravity will now be explored.

\subsection{Phase Separation}
There are numerous methods of phase separation. Some common phase separators use capillary forces or hydrophilic/hydrophobic systems. An example of one of these systems is the integral wick separator (IWS) which uses pours hydrophilic plates to allow energy transfer between a coolant line and humid air. Incoming air will cool, allowing humidity to condensate onto plates. The separator then utilizes capillary action to move the condensate from the plates to a volume filled with wicking material, ie. a material designed to take on moisture by encouraging capillary action. Condensate in the wicking material is then separated via an outlet line, which is maintained at a lower pressure to allow water removal and with a hydrophilic membrane cover to prevent gas from entering.\cite{ellis2013tangential}

One key issue with this, and many other designs of phase separators systems, is that they are typically designed for specific uses, ie they need redesigned if required for a different application. Thus, a change to the design of the system it is being combined with could well require a change in the design of the separator. This can be seen with capillary style separators, which are good for low volume flow but are completely impractical for high flow rate systems. It would be a great simplification to find a method of phase separation which can be easily used for many applications with little design change.

A key contemporary method of separation of gas and liquid in microgravity is by vortex separation, also known as cyclonic separation. Vortex separators have been found to be quite versatile in many applications. They can be easily integrated into systems with little change to the design of the separator so it is evident that they could be an interesting and fruitful area of research.\cite{bean2002vortex}

Using rotational movement, it is possible to separate mixtures of gases and liquids, or indeed any other two phase substance. A diagram of a vortex separator can be seen in \cref{fig:vortex}.

\begin{figure}[ht]
\centering
\includegraphics[scale=1.0]{images/vortex-sep.png}
\caption{Texas A \& M Vortex Separator}
\label{fig:vortex}
\end{figure}

The nozzle acts as an inlet letting a mixture of liquid and gas, ie a two phase substance, into the separator. The injection of fluid in via the nozzle will convert pressure head, the internal energy of a fluid due to the pressure exerted on its container, into velocity. The momentum of this injected fluid is coupled to the rotating cylinder thus forming a rotational flow. Due to this rotational movement, and hence radial acceleration, the denser liquid remains at the wall of the chamber and the less dense gas migrates radially inward.\cite{hawkes1998ultrasonic} 

This forms a vortex/cyclone along the centre axis of the cylinder. Under microgravity conditions, and neglecting capillary forces, the vortex will take the shape of a cylinder made up of gas surrounded by an annulus, doughnut shape, of dense liquid.
\todo[inline]{you need to refer to this image if you are going to use it}
\begin{figure}[ht]
\centering
\includegraphics[scale=0.6]{images/viewer.png}
\label{fig:vortex}
\caption{Top Down View of a Vortex Separator during Microgravity Testing}
\end{figure}

As it is in effect, buoyancy which controls the separation of dissimilar density fluids can be used to describe the force acting on the less dense fluid, ie the gas. The force acting on the infinitesimal unit of less dense fluid can be calculated using Newton’s second law shown in \cref{eq:nii}


\begin{equation}
mathrm{d\underline{F}=\underline{a}dm=\underline{a}\rho dV}
\label{eq:nii}
\end{equation}
Where $d\underline{F}$ the force acting on the infinitesimal unit of less dense fluid, $\underline{a}$ is the acceleration, $dm$ is the mass of the infinitesimal unit, $\rho$ the density of the fluid and $dV$ is the volume of the infinitesimal unit.

On earth the acceleration is gravity which allows a natural phase separation to occur. Since the vortex separator is for use in microgravity we must provide an alternative acceleration. This alternative is in the form of the radial acceleration provided by the radial movement of the fluid, which is created as discussed above.  The centripetal acceleration which is, in cylindrical coordinates is written as is shown in \cref{eq:cent_accel}

\begin{equation}
a_c(r,\theta, z)=\frac{v_c(r,\theta, z)^2}{r}
\label{eq:cent_accel}
\end{equation}

Where $r$, $\theta$, $z$ are the radial, angular and vertical positions respectively and is $v_c$ is the velocity.

By expressing the pressure on the inner side of an element of fluid
\begin{equation}
P_1=p -\frac{\partial p}{\partial r} \frac{d r}{2}
\label{eq:pressure1}
\end{equation}

Where $p$ is the pressure at the midpoint of the element and r is the outer side from the midpoint. Then 

By expressing the pressure on the inner side of an element of fluid as in \cref{eq:pressure2}
\begin{equation}
P_1=p+\frac{\partial p}{\partial r} \frac{d r}{2}
\label{eq:pressure2}
\end{equation}

Combining \cref{eq:pressure1} and \cref{eq:pressure2} yields an expression for the surface force in the radial direction, given by \cref{eq:surface_force}

By expressing the pressure on the inner side of an element of fluid as in \cref{eq:pressure2}. Similar expressions can be found in the z and $\theta$ directions.
\begin{equation}
d\underline{F}=-\frac{\partial p}{\partial r}r drd\theta{}dz\hat{r}
\label{eq:surface_force}
\end{equation}

The combination of these expressions gives \cref{eq:bouy} which describes the total resultant of the body and surface force, which must be zero. Thus a term can be obtained for the pressure field which provides a force necessary for buoyancy to occur:

\begin{equation}
p(r)=\frac{1}{2}p_l \omega^2 r^2 + p_o
\label{eq:bouy}
\end{equation}

Where $\omega$ is the averae angular velocity $p_o$ the initial pressure in the chamber and $p_l$ is the density of the more dense fluid, in this case the more dense fluid is the liquid.

As can be seen this pressure profile increases quadratically for a centripetal acceleration but only increases linearly for a gravitational system, constant acceleration. As gravity can be considered constant while centripetal acceleration of a fluid increases linearly with gradient, the pressure gradient developed by a rotating liquid flow increases from the central axis as the square of the radial position. This increased pressure means that the movement of bubbles in a vortex separator can be quite quick even at relatively low energies. Let us now inspect the movement of individual bubbles.

There is an effect on the buoyancy force acting on gas bubbles within the rotating flow. Consider the bubble shown in \cref{fig:bubble}

\todo[inline]{you need to refer to this image if you are going to use it}
\begin{figure}[ht]
\centering
\includegraphics[scale=1.0]{images/bubble.png}
\caption{A Bubble within a Vortex Separator}
\label{fig:bubble}
\end{figure}

The pressure gradient, \cref{eq:bouy} is in terms of radial location within the vortex separator. In order to determine the buoyancy force acting on the bubble shown above, \cref{fig:bubble},\cref{eq:bouy} must be translated into the coordinate system used to describe the bubble, ie. spherical polar. This works out as is shown in \cref{eq:bubble}

\begin{equation}
p(\alpha,\beta)=\frac{1}{2}p_l \omega^2\left[\sin^2{\beta}+L^2-2LR\sin^2{\beta}\cos^2{\alpha}\right]+p_o
\label{eq:bubble}
\end{equation}

With $\beta$ the angle in the median plan $\alpha$ the angle in the azimuthal plane and L the distance from the separator axis to the centre of the bubble.

This pressure can be interpreted as acting normal to the surface of the bubble and also acting inwards towards the centre. Due to spherical symmetry of the coordinate system used and the pressure only depending on the radial position, the angular and axial components of the pressure balance over the entire surface of the bubble. 

It can then be shown that in the cylindrical coordinate system of the separator this pressure acts in the radial direction only. Integration of the pressure over the surface of the bubble yields a buoyancy force in the form given by \cref{eq:bouy_force}

\begin{equation}
F_B=-p_l\omega^2 LV\hat{r}
\label{eq:bouy_force}
\end{equation}

Where $\omega$ is the angular velocity of the bubble and $V$ is the volume of the bubble. This buoyancy increases linearly with the radial position thus a bubble entering the vortex separator at the wall, a where the entrance nozzle is placed, will experience the greatest buoyancy force.



When the velocity of the bubble is calculated, it is found that the vortex separator is extremely effective with a bubble quickly separating from the bulk of the liquid.

(for a more detailed analysis of the forces acting within the separator one should consult \todo[inline]{I don't understand this}

A key benefit of vortex separators, in a microgravity environment, is the lack of moving parts, hence avoiding the problems with unreliability which can impact on mechanical systems. Also, in comparison with mechanical systems, the only power needed to operate a vortex separator is the pressure head. This small power consumption is important in as energy is limited while undertaking space travel. 
(Note that the above theory is demonstrated with the thought of a gas/liquid mixture in mind but can also be used for liquid/solid, such as for particulates in suspension, or gas/solid, such as for smoke).


\subsection{Fractionation of Particulates}

The vortex based separator is appropriate, and indeed very effective, for the separation of two phase substances. However, a second important separation process worth investigating is the fractionation of different sized particulates. This could be the separation of microbes from a product of a process undergone in a bioreactor or indeed the separation of particulates from suspension. By sorting microbes from product and waste it would be possible to recycle as many materials as possible. One such application already in use is the automatic change of nutrient solution and the harvesting of cell products in certain cell reactors. 

The exposing of particles to an ultrasonic standing wave causes them to move to pressure nodes or antinodes, ie particles become concentrated at half-wavelength intervals. In a plane stationary field particles will concentrate in uniform planar sheets each separated by a half-wavelength distance. The positioning of particles in such a structure is due to the acoustic radiation force and diffusion as shown in \cref{fig:acoust}

\begin{figure}[ht]
\centering
\includegraphics[scale=1.0]{images/frac.png}
\caption{articles Exposed to Acoustic Radiation Force}
\label{fig:acoust}
\end{figure}

A fractionation method which can be used for different populations of cells in a mixture is discussed in M. Kumar et als 2004 paper ‘Fractionation of Cell Mixtures Using Acoustic and Laminar Flow Fields’.\cite{kumar2005fractionation} This technology utilizes an ultrasonic field and a laminar flow field in orthogonal directions through a rectangular chamber. The flow carries the cells through a chamber while the ultrasonic field causes the cells to migrate to the mid plane of the chamber at rates related to their size and other physical properties. 

\begin{figure}[ht]
\centering
\includegraphics[scale=1.0]{images/separator.png}
\caption{Microchamber with Standing Wave Allowing Acoustic Fractionation of Particulates}
\label{fig:separator}
\end{figure}

\Cref{fig:separator} shows such a separator. As can be seen in a reflector and transducer, where the transducer creates a planar acoustic field, are placed at a half wavelength separation. This allows a nodal point to be positioned at the mid-point of the chamber and will force the walls of the chamber to be antinodes. AN inlet splitter restricts the lateral position of the particulates as they enter the chamber. From the left of the splitter a carrier stream is introduced which should be injected in such a way that it has a laminar flow throughout the channel when the acoustic transducer is not operating. 

Without the presence of the standing waves the particulates would simply travel down the chamber with the only force acting on them coming from the feed and carrier stream.
As long as the wavelength of this standing wave is large when compared to the size of the particles suspended the primary acoustic force, Fac , is given by \cref{eq:acoustic_force}


\begin{equation}
F_{ac}=4\pi R^3 \kappa E_{ac} F \sin{(2\kappa x)}
\label{eq:acoustic_force}
\end{equation}

Where $\kappa$ is the wave number of the applied sound wave, $E_{ac}$ is the energy per unit volume, R is the particle radius, x the position in the fluid relative to an acoustic pressure antinode, F is the acoustic contrast factor, which is given by \cref{eq:contrast}

\begin{equation}
F=\frac{1}{3}\left[\frac{5\rho_{\rho} - 2\rho_f}{\rho_f+2\rho_{\rho}}-\frac{\gamma_{\rho}}{\gamma_f}\right]
\label{eq:contrast}
\end{equation}

Where $\gamma_f$ is the compressibility of the fluid, $\gamma_p$ the compressibility of the particle, $\rho_f$ is the carrier fluid density and $\rho_{\rho}$ is the density of the particle  has been found experimentally, in the absence of exterior forces, those particles with a positive acoustic contrast factor move towards the pressure nodes and particles with a negative acoustic contrast factor move towards pressure nodes.
At the end of the rectangular chamber a flow splitter divides the cell suspension into two different streams. Hence the cells which respond faster to the acoustic radiation will be divided from the cells that respond slower. Also, the cells with a positive acoustic contrast factor will be separated from those with a negative. With a series of these it would be possible to separate a large number of particulates that have a range of different properties.

This is a relatively simple mechanism when used for a truly spherical particulate. However, when being applied to a cell suspensions more variables need to be taken into account.  Cells are somewhat smaller than polystyrene particles as had been previously used. This causes cells to have a susceptibility to smaller acoustic forces than polystyrene particles. Also cells may not be spherical and may tend to form clusters. The adaptation to suit these needs is discussed in M.Kumar, D.Feke and J.Belovich’s 2004 paper, ‘Fractionation of Cell Mixtures Using Acoustic and Laminar Flow Fields’ 

\section{Conclusion}
\bibliography{microbes}
\bibliographystyle{unsrt}

\end{document}
