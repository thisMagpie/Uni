\documentclass[]{report}   % list options between brackets
\usepackage[pdftex]{graphicx}
\usepackage{wrapfig}
\DeclareGraphicsExtensions{.png}




% type user-defined commands here

\begin{document}



\title{An investigation into the usefulness of the Chaos Game algorithm fo modelling the structure of the Y Chromosome }   % type title between braces
\author{Magdalen Berns}         % type author(s) between braces
\date{April 6, 2012}    % type date between braces
\maketitle
\chapter*{Introduction} 


\section{Introduction}     % section 1.1
\subsection{History}       % subsection 1.1.1

\chapter*{Object Oriented Methodolgy and Design} 

\section{Experiment 1}     % section 2.1
Where possible repeated code was avoided.The classes were catagorised into their respective tasks and in doing this it was
possible to design a Control.java class which had the bulk of the methods which needed to access the same coordinates:
To initialise the square and triangle polygon.
Check if a if a coordinate is inside the polygon
Select a random vertex and place a point inside.

 
\begin{equation}
 r * (x_{i}-x)*x
\end{equation}
made use of the contains() method in the Polygon class. 
 A lot of the code was dependent on coordinates so it was a difficult decision sometimes. 
	The DNA  files where fairly big (HOW BIG?) I was keen to avoid making the 	program work harder than it need to by using additional memory by instantiating variables in more classes than than necessary.
	To this end I structured the code.
	I used the polygon class from the awt package. This already had a lot of useful methods which helped with the project so I saw fit to use them. The main useful method on the package was “contains()” which returns true if a point is inside the polygon.
	I made my own Control.java class which took care of all methods which heavily depended on coordinates. I hard coded the width and height dimension of the shape but for the individual coordinates of the shape I made them scalable.  
	I made a random class 
	File reader class
	
	For this reason I was a bit counter intuitive as I did not create a class each for the square and triangles. 
	
\subsection{Usage}         % subsection 2.1.1
\chapter*{Results} 
          % chapter 2
\section{Experiment 1}     % section 2.1

	The random selection of points using a square and triangle polygon was treated as a seperate process altogether from those of the file. 
The reason for this was that it was preferable to have a single frame for both the triangle and square to compare their patterns using different angles.
This meant a Control.java class could be used for common coordinates and methods to the random processess involved in experiment 1.
Since the fraction for the dna file was a hard coded value of 0.5 and the data from the file would require exceptions to be thrown and a lengthy data storage process (SIZE OF File??) 
it was necessary it avoided 
o assess the behaviour of both the square and equalateral without too much user interaction. 
 A lot of the code was dependent on coordinates so it was a difficult decision sometimes. 
	The DNA  files where fairly big (HOW BIG?) I was keen to avoid making the 	program work harder than it need to by using additional memory by instantiating variables in more classes than than necessary.
	To this end I structured the code.
	I used the polygon class from the awt package. This already had a lot of useful methods which helped with the project so I saw fit to use them. The main useful method on the package was “contains()” which returns true if a point is inside the polygon.
	I made my own Control.java class which took care of all methods which heavily depended on coordinates. I hard coded the width and height dimension of the shape but for the individual coordinates of the shape I made them scalable.  
	I made a random class 
	File reader class

\begin{figure}[htb]
\includegraphics{/Home/s0946936/Desktop/CSlast/half.png}
\caption{(n=3,r=0.5) and (n=4, r=0.5) for equalateral triangle and square polygons using Chaos Game algorithm}
\label{fig:Key result of Experiment 1 (r=0.5)}
\end{figure}




\begin{wrapfigure}{3.5in}
%%%%%%%%%%%%%%%%%%%%%%%%%%%%%%%%%%%%%%%%%%%%%%%%%%%%%%%%%%%%%%%%%%%%%%%%%%%%%%%%%%%%%%%
%%% You will need to add \usepackage{wrapfig} to your preamble to use textwrapping %%%
%%%%%%%%%%%%%%%%%%%%%%%%%%%%%%%%%%%%%%%%%%%%%%%%%%%%%%%%%%%%%%%%%%%%%%%%%%%%%%%%%%%%%%%
 \centering
 \includegraphics[bb=0 0 587 257,keepaspectratio=true]{half.png}
 % half.png: 783x343 pixel, 96dpi, 20.71x9.07 cm, bb=0 0 587 257
 \caption{Key Resul; r=0.5}
\end{wrapfigure}


	
	. 
	
\subsection{Usage}         % subsection 2.1.1



\begin{thebibliography}{9}
  % type bibliography here
\end{thebibliography}

\end{document}