\documentclass{article}
\usepackage[svgnames]{xcolor}
\usepackage{pdfpages}
\usepackage{tikz}
\usepackage{animate}
\usepackage{amsmath}

\usepackage{hyperref} %% to turn the links clickable and load this before cleveref
\usepackage{cleveref}
\crefname{table}{table}{tables}
\Crefname{table}{Table}{Tables}
\crefname{figure}{figure}{figures}
\Crefname{figure}{Figure}{Figures}
\crefname{equation}{equation}{equations}
\Crefname{Equation}{Equation}{Equation}




\tikzset{
  every node/.style={
    anchor=mid west,
  }
}

\makeatletter
\pgfkeys{/form field/.code 2 args={\expandafter\global\expandafter\def\csname field@#1\expandafter\endcsname\expandafter{#2}}}

\newcommand{\place}[3][]{\node[#1] at (#2) {\csname field@#3\endcsname};}
\makeatother

\newcommand{\xmark}[1]{\node at (#1) {X};}



\begin{document}

\foreach \mykey/\myvalue in {
  ctsfn/{This is from tutorial question 2.2},
  metsp/{This is in a tutorial},
} {
  \pgfkeys{/form field={\mykey}{\myvalue}}
}

\includepdf[
  pages=2,
  picturecommand={%
    \begin{tikzpicture}[remember picture,overlay]
\tikzset{every node/.append style={fill=Honeydew,font=\large}}
\place[name=ctsfn]{0cm,22cm}{ctsfn}
\place[name=metsp]{1cm,13cm}{metsp}
\draw[ultra thick,blue,->] (ctsfn) to[out=120,in=90] (5cm,17.3cm);
\draw[ultra thick,blue,->] (metsp) to[out=130,in=60] (4cm,14cm);
    \end{tikzpicture}
  }
]{diffraction2009.pdf}
\section*{A.1}

For a cavity of reflectivity $R$ half height intensity given by \cref{eq:intensityhalf} is shown in \cref{eq:reflectivityhalf}

\begin{equation}
\label{eq:intensityhalf}
I=\frac{I_{0}} {2}
\end{equation}

\begin{equation}
\label{eq:reflectivityhalf}
\frac{I_{0}} {2}=\frac{I_{0}(1-R)^{2}} {(1-R)^{2}+4Rsin^{2}(\frac{\delta}{2})}\implies (1-R)^{2} =\frac{1}{2}(1-R)^{2} +4R\mathrm{sin^{2}\left(\frac{\delta}{2}\right)}
\end{equation}

In this problem we are given that $R=0.9$ so if we substitute this value in we can solve for $\delta$ 

\begin{equation}
\label{eq:delta}
1=\frac{4R\mathrm{sin^{2}\left(\frac{\delta}{2}\right)}}{(1-R)^{2}}\implies \mathrm{sin^{2}{\left(\frac{\delta}{2}\right)}}=\frac{(1-R)^{2}}{4R}\implies \mathrm{sin{\left(\frac{\delta}{2}\right)}}=\frac{(1-R)}{2\sqrt{R}}
\end{equation}
If we assume that the peak is narrow this allows us to perform a small angle approximation so that $\mathrm{sin}\left(\frac{\delta}{2}\right)\approx\left(\frac{\delta}{2}\right)$. Therefore our final expression for delta is given by \cref{eq:deltafin}

\begin{equation}
\label{eq:deltafin}
\delta=\frac{(1-R)}{\sqrt{R}}=\frac{(1-0.9)}{\sqrt{0.9}}=0.1054
\end{equation}

Which we can us to determine the wavelength from this expression given in the question with $n=1$ and $d=1$ and the answer is 
shown in \cref{eq:deltawav} which gives the full wavelength of the monoquasichomatic light source at half height FWH

\begin{equation}
\label{eq:deltawav}
\delta=\frac{4\pi d n }{\lambda}\implies \lambda=\frac{4\pi d n }{\delta}= \lambda=\frac{4\pi }{0.1054}=119.22\mathrm{m}
\end{equation}


%%%%%%%%%%%%%%%%%%%%% A2 %%%%%%%%%%%%%%%%%%
\section*{A.2}
The Rayleigh limit of angular resolution is shown \cref{eq:rayleigh}

\begin{table}[htdp]
\caption{default}
\begin{center}
\begin{tabular}{|c|c|}
 & n \\
\hline
$\mathrm{Mg F_{2}}$ & 1.38 \\
$\mathrm{Al_{2}O_{3}}$ & 1.62

\end{tabular}
\end{center}
\label{default}
\end{table}%

\begin{equation}
\label{eq:rayleigh}
\Delta \theta_{0}=\frac{1.22 D}{\lambda}
\end{equation}



\includepdf[
  pages=3,
  picturecommand={%
    \begin{tikzpicture}[remember picture,overlay]
\tikzset{every node/.append style={fill=Honeydew,font=\large}}
\place[name=ctsfn]{11cm,10cm}{ctsfn}
\place[name=metsp]{11cm,9cm}{metsp}
\draw[ultra thick,blue,->] (ctsfn) to[out=135,in=90] (9cm,17.3cm);
\draw[ultra thick,blue,->] (metsp) to[out=155,in=70] (6cm,9cm);
    \end{tikzpicture}
  }
]{diffraction2009.pdf}

\end{document}