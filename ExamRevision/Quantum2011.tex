\documentclass[12pt]{article}
\usepackage{amsthm} 

\usepackage{amsmath}
\usepackage{amsfonts}
\usepackage{amssymb}
\usepackage[hmargin=1.5cm,vmargin=1.5cm]{geometry}
\usepackage[super,comma,sort&compress, square]{natbib}
\usepackage{graphicx}
\usepackage{tikz}
\usepackage{pgfplots}
\usepackage{float}
\usepackage{csvsimple}
\usepackage{gnuplottex}
\usepackage{todonotes}
\usepackage{csquotes}
\usepackage{pgfkeys}


\pgfplotsset{compat=1.7}
\usepackage[english]{babel}
\pgfplotsset{every axis y label/.append style={xshift=1.5em}}



\usepackage{hyperref} %% to turn the links clickable and load this before cleveref
\usepackage{cleveref}
\crefname{table}{table}{tables}
\Crefname{table}{Table}{Tables}
\crefname{figure}{figure}{figures}
\Crefname{figure}{Figure}{Figures}
\crefname{equation}{equation}{equations}
\Crefname{Equation}{Equation}{Equation}


\title{Quantum Mechanics Solutions to May 2011}
\author{}
\date{\today}

\begin{document}
\maketitle

\section*{A.1}
\noindent
\textbf{The Components of angular momentum are given by: $$\hat{L_{j}}\epsilon_{ikj}\hat{x_{k}}\hat{p_{l}}$$ Where the indices $j,k,l$ range from 1-3 and summation over a repeated series is understood. Write an explicit expression for $L_{1}$}
\subsection*{Answer i.}
\begin{equation}
\label{eq:ang1}
L_{1}=\sum^{3}_{k,l}\epsilon_{j,k,l}\hat{x_{l}}\hat{p_{l}}=\epsilon_{1,2,3}\hat{x_{2}}\hat{p_{3}}+\epsilon_{1,3,2}\hat{x_{3}}\hat{p_{2}} 
=\hat{x_{2}}\hat{p_{3}}-\hat{x_{3}}\hat{p_{2}}
\end{equation}
\subsection*{}
\textbf{From the canonical relations $$[\hat{x_{j}},\hat{x_{k}}]=[\hat{p_{j}},\hat{p_{k}}]=0$$
 $$[\hat{x_{j}},\hat{p_{k}}]=\delta_{j,k}=1,2,3$$ \noindent
 prove that $[L_{1},L_{2}]=L_{3}$}

\subsection*{Answer ii.}
\noindent
From \cref{eq:ang1} it follows that \cref{eq:ang2} is also true. 


\begin{equation}
\label{eq:ang2}
L_{2}=\hat{x_{3}}\hat{p_{1}}-\hat{x_{1}}\hat{p_{3}}
\end{equation}

\begin{equation}
\label{eq:ang3}
[L_{1},L_{2}]=[\hat{x_{2}}\hat{p_{3}}-\hat{x_{3}}\hat{p_{2}},\hat{x_{3}}\hat{p_{1}}-\hat{x_{1}}\hat{p_{3}}]=
[\hat{x_{2}}\hat{p_{3}},\hat{x_{3}}\hat{p_{1}}]-[\hat{x_{1}}\hat{p_{3}},\hat{x_{3}}\hat{p_{3}}]-[\hat{x_{3}}\hat{p_{2}},\hat{x_{3}}\hat{p_{1}}]+
[\hat{x_{3}}\hat{p_{2}},\hat{x_{1}}\hat{p_{3}}]
\end{equation}
\noindent
We can use the identity $[A,B,C]=A[B,C]+[A,C]B$ on \cref{eq:ang3} to get the result which is shown in \cref{eq:ang4}


\begin{equation}
\label{eq:ang4}
\hat{x_{2}}[\hat{p_{3}},\hat{x_{3}}]\hat{p_{1}}+\hat{x_{1}}[\hat{p_{3}},\hat{x_{3}}]\hat{p_{2}}=
i\hbar[\hat{x_{1}}\hat{p_{2}}-\hat{x_{2}}\hat{p_{1}}]=i\hbar{\hat{L_{3}}}
\end{equation}

\subsection*{}
\textbf{Write the generic expression for the commutator $[L_{j},L_{k}]$} 

\begin{equation}
\label{eq:sol1}
[L_{j},L_{k}]=i\hbar{\epsilon_{j,k,m}\hat{L_{m}}}
\end{equation}

\clearpage
\section*{A.2}
\noindent
\textbf{The creation and annihilation operators for a harmonic oscillator of mass, m and frequency $\omega$ are respectively given by 
\cref{eq:creation} \cref{eq:annihilation}. Compute the commutator $[a,a^{+}]$}

\begin{equation}
\label{eq:creation}
\hat{a}=\frac{m\omega}{2\hbar}\hat{x}+i\sqrt{\frac{1}{2m\omega\hbar}}\hat{p}
\end{equation}


\begin{equation}
\label{eq:annihilation}
\hat{a^{+}}=\frac{m\omega}{2\hbar}\hat{x}-i\sqrt{\frac{1}{2m\omega\hbar}}\hat{p}
\end{equation}

\subsection*{Answer i.}

\begin{equation}
\label{eq:com1}
[a,a^{+}]=\left[\frac{m\omega}{2\hbar}\hat{x}+i\sqrt{\frac{1}{2m\omega\hbar}}\hat{p},\frac{m\omega}{2\hbar}\hat{x}
-i\sqrt{\frac{1}{2m\omega\hbar}}\hat{p}\right]
\end{equation}

\begin{equation}
\label{eq:com2}
[a,a^{+}]= \frac{i}{2\hbar} [\hat{p},\hat{x}]-[\hat{x},\hat{p}]=\frac{i}{2\hbar}[-i\hbar-i\hbar]=1
\end{equation}
\noindent
\textbf{ Invert \cref{eq:creation,eq:annihilation} to find the momentum operator  $\hat{p}$}
\subsection*{Answer ii.}
\begin{equation}
\label{eq:com3}
a-a^{+}= \left(\frac{m\omega}{2\hbar}\hat{x}+i\sqrt{\frac{1}{2m\omega\hbar}}\hat{p}\right)-
\left(\frac{m\omega}{2\hbar}\hat{x}+i\sqrt{\frac{1}{2m\omega\hbar}}\hat{p}\right)=i\sqrt{\frac{1}{m\omega\hbar}}\hat{p}
\end{equation}

\begin{equation}
\label{eq:sol2}
\hat{p}=i \sqrt{m \omega \hbar } (a-a^{+})
\end{equation}


\begin{equation}
\label{eq:sol3}
\hat{p}=-i \sqrt{m \omega \hbar } (a^{+}-a)
\end{equation}
\noindent
Where $$i(a-a^{+})=-i(a^{+}-a)$$
\noindent
\textbf{What is an expectation value of $\hat{p}$ in a stationary state of the Hamiltonian?}

\subsection*{Answer iii.}
\begin{equation}
\label{eq:expect}
\langle \psi | \hat{p} | \psi \rangle = -i \sqrt{m \omega \hbar } \langle \psi | (a^{+}-a) | \psi \rangle  = 0 
\end{equation}
\clearpage
\section*{A3}
\textbf{The three components of the angular momentum operator,
 $\hat{L}^{2}=\hat{L}^{2}_{x}+\hat{L}^{2}_{y}+\hat{L}^{2}_{z}$ be the square of the angular momentum operator and $| l,m\rangle$ denote the simultaneous 
 eigenstates. of $L_{z}$ and $\hat{L}$ with respective eigenvalues $\hbar^{2}l(l+1)$ and $\hbar m$ let us also define
  $\hat{L}_{\pm}=\hat{L}_{x}\pm i L_{y}$. }
  \noindent
\textbf{Show that }
 $l_{+} | l,m\rangle=c_{+} | l,m+1\rangle$ \textbf{Where $c_{+}$ is a normalisation constant.}
 
 \subsection*{Answer i.}
 
 \clearpage 
 \section*{A4}
 Consider two particles with respective angular momentum $j_{1}$ and $j_{2}$. What are the possible values for the total angular momentum 

\subsection*{Answer}
 
\begin{equation}
\label{eq:mom1}
J=j_{1}+j_{2},j_{1}+j^{-1}_{2},...|j_{1}-j_{2}|
\end{equation}

\begin{equation}
\label{eq:mom2}
M=-J=-J+1,...,+J
\end{equation}
\clearpage

%%%%%%%%%%%%%%% 
%% B1 
%%%%%%%%%%%%%%%

\section*{Question B.1}
\noindent
\textbf{Consider a system of mass $\mu$}

\begin{equation}
\label{eq:sol4}
V(r) = \left\{
  \begin{array}{l l }
    \infty, & \quad \text{if r  $<$ R}\\
    0 & \quad \text{if } R < r < R+a \\
      -\infty & \quad \text{if } R+a < r\\

  \end{array} \right.\end{equation}
\noindent  
\textbf{In order to find the stationary states we factorise the wave function in spherical polar coordinates.}

\begin{equation}
\label{eq:wavefn}
\psi=R(r)Y_{l}^{m}(\theta,\phi)\end{equation}
\noindent
\textbf{Where $Y_{l}^{m}$ are spherical harmonics.}

\paragraph
\noindent
Physical interpretation: This is a particle confined to a spherical shell.

\subsection*{Question B1 a)}
\textbf{Let $L_{x},L_{y},L_{z}$ respectively denote the $x,y$ and $z$ components of angular momentum. 
The norm of angular momentum is} $$\hat{L}^{2}=\hat{L}^{2}_{x}+\hat{L}^{2}_{y}+\hat{L}^{2}_{z}$$ 
\noindent
\textbf{Write down the action of $\hat{L}_{x}$ and  $\hat{L}^{2}$ and identify the eigenvalue equations and check the dimensions}
\subsection*{Answer B1 a)}

$$\hat{L}_{z}Y_{l}^{m}(\theta,\phi)=\hbar m Y_{l}^{m}(\theta,\phi)$$
$$\hat{L}^{2}Y_{l}^{m}(\theta,\phi)=\hbar^{2}l(l+1)Y_{l}^{m}(\theta,\phi)$$

Where $\hbar m$ and $\hbar^{2}l(l+1)$ are the eigenvalues of $\hat{L}_{z}$ and $\hat{L}^{2}$ respectively.

To check the dimensions it is important to remember that $\hbar$ has the dimensions of angular momentum as expressd in \cref{dimensions1}

\begin{equation}
\label{eq:dimensions1}
\hbar_{dimensions}=ML^{2}T^{-1}  
\end{equation}
\clearpage
\subsection*{Question B1 b) i.}
\textbf{The laplacian in three dimensions can be written as}

\begin{equation}
\label{eq:lap}
\nabla^{2}=\frac{1}{r^{2}}\frac{\partial}{\partial{r}} \left(r^{2 }\frac{\partial}{\partial{r}}\right)-\frac{\hat{L^{2}}}{\hbar^{2} r^{2}}
\end{equation}
\noindent
\textbf{Check the dimensions on both sides of this equation.}

\subsection*{Answer B1 b) i.}
$$[r]=L$$

$$\left[\frac{\partial}{\partial{r}}\right]=L^{-1}$$

$$\text{LHS: }\nabla^{2}=L^{-2}$$

$$\text{RHS: }L^{-2}L^{-1}L^{2}L^{-1}-L^{-2}=L^{-4}L^{2}-L^{-2}=L^{-2}=\text{LHS}$$
\clearpage
\subsection*{Question B1 b) ii.}

\textbf{Check it can be solved by separation of variables.}

\subsection*{Answer B1 b) ii.}

\begin{equation}
\label{eq:egg}
\hat{H}\psi=\hat{E}\psi
\end{equation}
\noindent

\begin{equation}
\label{eq:ham}
\hat{H}=\frac{\hat{p}^{2}}{2\mu}+V(r)=\frac{\hbar^{2}}{2\mu}\nabla^{2}+V(r)
\end{equation}

\noindent
Substituion of \cref{eq:lap} into \cref{eq:ham} yields
\begin{equation}
\label{eq:hamlap}
\hat{H}=\frac{\hbar^{2}}{2\mu}{\left(\frac{1}{r^{2}}\frac{\partial}{\partial{r}} \left(r^{2 }\frac{\partial}{\partial{r}}\right)-
\frac{\hat{L^{2}}}{\hbar^{2} r^{2}}\right)^{2}}+V(r)
\end{equation}
\noindent
Therefore if we use the Hamiltonian operator to operate on the wave function as expressed in \cref{eq:wavefn} 

\begin{equation}
\label{eq:hamlap}
\hat{E} | \psi  \rangle =\hat{E} | R(r)Y_{l}^{m}(\theta,\phi)  \rangle =\left[\frac{\hbar^{2}}{2\mu}{\left(\frac{1}{r^{2}}\frac{\partial}{\partial{r}} \left(r^{2 }\frac{\partial}{\partial{r}}\right)-
\frac{\hat{L^{2}}}{\hbar^{2} r^{2}}\right)^{2}}+V(r)\right]R(r)Y_{l}^{m}(\theta,\phi)
\end{equation}


\begin{equation}
\label{eq:hamlapr}
\hat{E} | \psi  \rangle =\hat{E} | R(r))  \rangle =\left[\frac{\hbar^{2}}{2\mu}{\left(\frac{1}{r^{2}}\frac{\partial}{\partial{r}} \left(r^{2 }\frac{\partial}{\partial{r}}\right)-
\frac{l(l+1)}{r^{2}}\right)^{2}}+V(r)\right]R(r)
\end{equation}

Let $R(r)=\frac{x(r)}{r}$ and applying the operation shown in \cref{eq:subs}:

\begin{equation}
\label{eq:subs}
\frac{\partial}{\partial r}\left(\frac{x(r)}{r}\right)=\frac{x(r)'}{r}-\frac{x(r)}{r^{2}}
\end{equation}


\begin{equation}
\label{eq:subs2}
\frac{\partial}{\partial r}r^{2}\left(\frac{\partial}{\partial r}\right)=x'+rx''-x'\implies 
\frac{1}{r^{2}}\frac{\partial}{\partial r}r^{2}\left(\frac{\partial}{\partial r}\right)=\frac{x''(r)}{r}
\end{equation}

\begin{tikzpicture}
    \begin{axis}[no marks,
        yticklabel style={/pgf/number format/fixed},
        %  changes tick labels to a number instead
        %  of exponential notation:
        yticklabel={%
            \pgfmathfloatparsenumber{\tick}%
}, xticklabel={%
            \pgfmathfloatparsenumber{\tick}%
},]
        \addplot {exp(-x)};
\end{axis}
\end{tikzpicture}

Substitute into 
\begin{equation}
\label{eq:exp}
\frac{\hbar^{2}}{2 \mu} \left(\frac{x''(r)}{r}-\frac{l(l+1)}{r^{2}}\right)+\frac{V(r)x(r)}{r}=\hat{E}\psi
\end{equation}
\clearpage
\subsection*{Question c)} 
\textbf{Let us define the function $\chi(r)=rR(r)$. Write down the radial part of the time independent Schrodinger equation as a 1-D differential equation for $\chi$.
Write down the boundary conditions for $\chi$. Give a physical interpretation for the boundary conditions.}

Substitute into 
\begin{equation}
\label{eq:exp2}
\frac{\hbar^{2}}{2 \mu}\chi^{2}(r)-\frac{\hbar l(l+1)}{2\mu}\chi(r)+V(r)\chi(r)=\frac{\hbar^{2}}{2 \mu}\chi^{2}-V_{eff}\chi=\hat{E}\psi
\end{equation}


\begin{equation}
\label{eq:}
\chi{(R)}=\chi{(R+a)}=0\implies R(R)=\frac{\chi(R)}{R}=0\implies \frac{\chi(R+a)}{R+a}=0
\end{equation}

The physical interpretation is that the particle is trapped. Planar well with $ \to \infty$

\subsection*{Question B1 c)}
\textbf{Solve the equation for $l=0$}

\begin{equation}
\label{eq:lo}
V_{eff}=\frac{l(l+1)}{r^{2}}+\frac{V(r)x(r)}{r}
\end{equation}

\noindent
\cref{eq:lo} implies that when $l=0$ the equation in \cref{eq:exp2} takes the form

\begin{equation}
\label{eq:exp3}
\hat{E}\chi=\frac{\hbar^{2}}{2 \mu}\chi^{2}+V(r)\chi
\end{equation}

\noindent
So at the boundary we can put the solution in the form 

\begin{equation}
\label{eq:soll1}
\chi=Ae^{ikr}+Be^{-ikr}
\end{equation}


\begin{equation}
\label{eq:soll2}
B=-Ae^{2ikR}
\end{equation}

\noindent
With $k=\frac{\sqrt{2\mu E}}{\hbar^{2}}$. Substituting \cref{eq:soll2} into \cref{eq:soll1}

\begin{equation}
\label{eq:soll3}
\chi=Ae^{ikr}+Ae^{2ikR}e^{-ikr}=Ae^{ikR}\left[  e^{ik(r-R)}+Ae^{-ik(r-R)} \right]
\end{equation}

\begin{equation}
\label{eq:soll4}
\chi=Ae^{ikR}\mathrm{sin({k(r-R))}}
\end{equation}
$\lambda(k+a)=0\text{ }\mathrm{sin}(ka)=0\text{ }ka=n\pi$

\subsection*{Question d)}
\noindent
\textbf{Show that the energy level between the ground state and the first exited state is $\Delta E=\frac{\hbar}{mR^{2}}$}

For $l=1,a<<R.R<r<R+a$

 \end{document}